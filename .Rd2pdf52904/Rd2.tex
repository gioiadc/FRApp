\nonstopmode{}
\documentclass[a4paper]{book}
\usepackage[times,inconsolata,hyper]{Rd}
\usepackage{makeidx}
\makeatletter\@ifl@t@r\fmtversion{2018/04/01}{}{\usepackage[utf8]{inputenc}}\makeatother
% \usepackage{graphicx} % @USE GRAPHICX@
\makeindex{}
\begin{document}
\chapter*{}
\begin{center}
{\textbf{\huge Package `FRApp'}}
\par\bigskip{\large \today}
\end{center}
\ifthenelse{\boolean{Rd@use@hyper}}{\hypersetup{pdftitle = {FRApp: FRAP Data Analysis Using Nonlinear Mixed Effect Models with Shiny}}}{}
\ifthenelse{\boolean{Rd@use@hyper}}{\hypersetup{pdfauthor = {Gioia Di Credico}}}{}
\begin{description}
\raggedright{}
\item[Type]\AsIs{Package}
\item[Title]\AsIs{FRAP Data Analysis Using Nonlinear Mixed Effect Models with Shiny}
\item[Version]\AsIs{1.0.0}
\item[Description]\AsIs{Analysis of Fluorescence Recovery After Photobleaching (FRAP) 
experiments using nonlinear mixed-effects regression models and analysis of 
the results. FRApp is not limited to the analysis of FRAP experiments only. 
Any nonlinear mixed-effects models with an asymptotic exponential functional 
relationship to hierarchical data in various domains can be fitted. 
The analysis of data available in the package is presented in Di Credico, G., 
Pelucchi, S., Pauli, F. et al. (2025) <}\Rhref{https://doi.org/10.1038/s41598-025-87154-w}{doi:10.1038/s41598-025-87154-w}\AsIs{>.}
\item[License]\AsIs{GPL (>= 3)}
\item[Encoding]\AsIs{UTF-8}
\item[LazyData]\AsIs{true}
\item[Imports]\AsIs{nlme, shiny, gplots, grDevices, graphics, methods, stats, utils}
\item[Roxygen]\AsIs{list(markdown = TRUE)}
\item[RoxygenNote]\AsIs{7.3.2}
\item[NeedsCompilation]\AsIs{no}
\item[Suggests]\AsIs{knitr,
rmarkdown,
testthat (>= 3.0.0)}
\item[Config/testthat/edition]\AsIs{3}
\item[Depends]\AsIs{R (>= 2.10)}
\item[VignetteBuilder]\AsIs{knitr, rmarkdown}
\item[URL]\AsIs{}\url{https://github.com/gioiadc/FRApp}\AsIs{}
\item[BugReports]\AsIs{}\url{https://github.com/gioiadc/FRApp/issues}\AsIs{}
\end{description}
\Rdcontents{Contents}
\HeaderA{FRAPdata}{FRAP analysis dataset}{FRAPdata}
\keyword{datasets}{FRAPdata}
%
\begin{Description}
The dataset provides 65 measures of fluorescence intensity equally spaced in time over a period of 100 seconds for 132 dendritic spines, for a total of 8580 observations.
Data presents a nested hierarchical structure: spines belongs to 54 neurons, grouped in 6 cultures.
The number of spines within neurons, and of neurons within cultures is not constant.
Experimental condition is applied to 28 neurons (CAP2-shRNA, 71 spines), whereas 26 are controls (SCR, 61 spines).
The dataset is a subset of the original version where spines with fluorescence intensity larger than 0.6 at time 0 have been excluded.
For a comprehensive description of the data and of the data cleaning steps, please refer to the paper below.
\end{Description}
%
\begin{Usage}
\begin{verbatim}
FRAPdata
\end{verbatim}
\end{Usage}
%
\begin{Format}
%
\begin{SubSection}{\code{FRAPdata}}

A data frame with 8580 rows and 6 columns:
\begin{description}

\item[Culture.id] Culture id, from 1 to 6
\item[genetic.id] Experimental condition, SCR for control and SH for CAP2-shRNA
\item[neuron] Neuron id and experimental condition, nested within cultures
\item[spine.ID] Spine id, nested within neurons
\item[time] Time, from 0 to 99.84
\item[y] Fluorescence intensities

\end{description}

\end{SubSection}

\end{Format}
%
\begin{Source}
Di Credico, G., Pelucchi, S., Pauli, F. et al. Nonlinear mixed-effects models to analyze actin dynamics in dendritic spines. Sci Rep 15, 5790 (2025). https://doi.org/10.1038/s41598-025-87154-w
\end{Source}
\HeaderA{FRApp}{FRApp}{FRApp}
%
\begin{Description}
A Shiny app to fit the nonlinear mixed effects model to FRAP data. By default,
R opens the Shiny app in the default browser.
\end{Description}
%
\begin{Usage}
\begin{verbatim}
FRApp(...)
\end{verbatim}
\end{Usage}
%
\begin{Arguments}
\begin{ldescription}
\item[\code{...}] shinyApp function options
\end{ldescription}
\end{Arguments}
%
\begin{Details}
FRApp is shiny app that provides a user-friendly interactive interface to
streamline the data analysis derived by fitting nonlinear mixed-effects
regression models with an asymptotic exponential functional relationship
on data with a hierarchical structure.

The application accepts the csv files format. By default, it uses the
semicolon as the field separator and the period as the decimal separator,
but you can select different separators from the drop-down menu.

The first line of the file must contain the variable names.
The Browse button allows you to load your own data file to be analyzed.

Via the application you can: estimate and compare exponential mixed-effects
models; print a model report; export the data and the model corresponding
objects.

The vignette describes how to reproduce the analysis presented in
Di Credico, G., Pelucchi, S., Pauli, F. et al. in Scientific Report
15, 5790 (2025) \url{doi:10.1038/s41598-025-87154-w}
\end{Details}
%
\begin{Value}
A list of objects that defines the Shiny app.

None
\end{Value}
%
\begin{Examples}
\begin{ExampleCode}
# Run the app
if (interactive()) {
  FRApp()
}

\end{ExampleCode}
\end{Examples}
\printindex{}
\end{document}
